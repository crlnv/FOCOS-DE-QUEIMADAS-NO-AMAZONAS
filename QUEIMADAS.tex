\documentclass{beamer}
\usepackage[utf8]{inputenc}
\usepackage{listings}
\usepackage{listings}
\usepackage[svgnames]{xcolor}
\usepackage{lstautogobble} 
\usepackage{graphicx}

\lstset{
  language=R,               % Definir linguagem para o código (opcional)
  frame=single,             % Colocar moldura ao redor do código
  numbers=left,             % Números à esquerda do código
  basicstyle=\ttfamily\small, % Fonte monoespaçada
  keywordstyle=\color{blue}, % Estilo para palavras-chave
  commentstyle=\color{gray}, % Estilo para comentários
  stringstyle=\color{orange} % Estilo para strings
}




\title{QUEIMADAS NO AMAZONAS}
\subtitle{UM MODELO COM COMPONENTE SAZONAL}
\author{Caroline Vasconcelos}


\usetheme{lucid}

\begin{document}
\frame{
 \titlepage
}
\frame {
 \frametitle{Contextualizando...}
\begin{figure}
        \centering
        % Primeira linha de imagens
        \includegraphics[width=0.45\linewidth]{imagens/jornal1.png} 
        \hfill
        \includegraphics[width=0.45\linewidth]{imagens/jornal2.png} 
        \\ % Quebra de linha
        % Segunda linha de imagens
        \includegraphics[width=0.45\linewidth]{imagens/jornal3.png} 
        \hfill
        \includegraphics[width=0.45\linewidth]{imagens/jornal4.png} 
    \end{figure}
 
}


\frame {
 \frametitle{Dados}
 \begin{itemize}
     \item Série: 1999-jan até 2023-dez; 
     \item Série mensal;
     \item Fonte: Portal de Queimadas, INPE.
 \end{itemize}

 \begin{figure}
     \centering
     \includegraphics[width=0.9\linewidth]{imagens/graf1.png}
     \caption{Série temporal dos focos de queimadas no Amazonas (1999-2023)}
     \label{fig:enter-label}
 \end{figure}
}

\frame {
 \frametitle{Dados}
 Para eliminar a \textbf{heterocedasticidade}, tomamos o \textbf{log}, e então
  \begin{figure}
     \centering
     \includegraphics[width=0.9\linewidth]{imagens/graf2.png}
     \caption{Log da série}
     \label{fig:enter-label}
 \end{figure}
}

\frame {
 \frametitle{Monthplot}
 A série exibe um \textbf{padrão sazonal}?
  \begin{figure}
     \centering
     \includegraphics[width=0.9\linewidth]{imagens/graf3.png}
     \caption{Comportamento mensal}
     \label{fig:enter-label}
 \end{figure}
}

\frame {
 \frametitle{Periodograma}
 Verificamos que o periodograma apresenta um pico e esse pico tem período 6.
  \begin{figure}
     \centering
     \includegraphics[width=0.9\linewidth]{imagens/graf4.png}
     \caption{Pico da série}
     \label{fig:enter-label}
 \end{figure}
}

\frame {
 \frametitle{ACF}
 ACF sugere um processo MA(5).
  \begin{figure}
     \centering
     \includegraphics[width=0.9\linewidth]{imagens/graf5.png}
     \caption{ACF da série}
     \label{fig:enter-label}
 \end{figure}
}

\frame {
 \frametitle{PACF}
 PACF sugere um processo AR(3).
  \begin{figure}
     \centering
     \includegraphics[width=0.9\linewidth]{imagens/graf6.png}
     \caption{PACF da série}
     \label{fig:enter-label}
 \end{figure}
}

\frame {
 \frametitle{USANDO A FUNÇÃO AUTO.ARIMA}

\begin{itemize}
     \item Modelo sugerido: ARIMA(3,0,5)(0,1,1); 
     \item AIC = 536.86;
     \item Resíduos
 \end{itemize}

  \begin{figure}
     \centering
     \includegraphics[width=0.9\linewidth]{imagens/graf7.png}
     \caption{Resíduos do modelo ARIMA(3,0,5)(0,1,1) }
     \label{fig:enter-label}
 \end{figure}
}

\frame {
 \frametitle{AUTO.ARIMA}
 ACF

  \begin{figure}
     \centering
     \includegraphics[width=0.9\linewidth]{imagens/graf8.png}
     \caption{ACF do modelo ARIMA(3,0,5)(0,1,1) }
     \label{fig:enter-label}
 \end{figure}
}

\frame {
 \frametitle{AUTO.ARIMA}
 PACF

  \begin{figure}
     \centering
     \includegraphics[width=0.9\linewidth]{imagens/graf9.png}
     \caption{PACF do modelo ARIMA(3,0,5)(0,1,1) }
     \label{fig:enter-label}
 \end{figure}
}

\frame {
 \frametitle{SARIMA (3,0,5)(0,2,2)}
 \begin{itemize}
     \item Para incorporar o comportamento de sazonalidade, utilizam-se os modelos ARIMA sazonais multiplicativos (SARIMA) (Morettin & Toloi, 2004);
     \item 2 diferenças;
     \item ACF
 \end{itemize}

\begin{figure}
     \centering
     \includegraphics[width=0.9\linewidth]{imagens/graf10.png}
     \caption{ACF do modelo SARIMA (3,0,5)(0,2,2)}
     \label{fig:enter-label}
 \end{figure}
}

\frame {
 \frametitle{SARIMA (3,0,5)(0,2,2)}
PACF
  \begin{figure}
     \centering
     \includegraphics[width=0.9\linewidth]{imagens/graf11.png}
     \caption{PACF do modelo SARIMA (3,0,5)(0,2,2)}
     \label{fig:enter-label}
 \end{figure}
}

\begin{frame}[fragile] 
 \frametitle{SARIMA (3,0,5)(0,2,2)}

Box-Pierce Test

\begin{lstlisting}[language=R]
X-squared = 0.086041, df = 1, p-value = 0.7693
\end{lstlisting}

  \begin{figure}
     \centering
     \includegraphics[width=0.9\linewidth]{imagens/graf12.png}
     \caption{Resíduos do modelo SARIMA (3,0,5)(0,2,2)}
     \label{fig:enter-label}
  \end{figure}
\end{frame}


\begin{frame}[fragile] 
 \frametitle{SARIMA (3,0,5)(0,2,2)}

Shapiro-Wilk Normality Test

\begin{lstlisting}[language=R]
W = 0.9858, p-value = 0.01019
\end{lstlisting}

  \begin{figure}
     \centering
     \includegraphics[width=0.9\linewidth]{imagens/graf13.png}
     \caption{Histograma dos resíduos do modelo SARIMA (3,0,5)(0,2,2)}
     \label{fig:enter-label}
  \end{figure}
\end{frame}


\frame{
  \frametitle{PREVISÃO}

 \begin{figure}
     \centering
     \includegraphics[width=0.9\linewidth]{imagens/graf14.png}
     \caption{Série incluindo previsão do modelo SARIMA (3,0,5)(0,2,2)}
     \label{fig:enter-label}
  \end{figure}
  
\end{frame}
}


\frame{
  \frametitle{MÉDIAS MENSAIS}

 \begin{figure}
     \centering
     \includegraphics[width=0.9\linewidth]{imagens/tab1.png}
     \caption{Previsão das médias mensais do modelo SARIMA (3,0,5)(0,2,2)}
     \label{fig:enter-label}
  \end{figure}
  
\end{frame}
}

 
\end{document}